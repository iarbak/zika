\documentclass{article}
\usepackage[utf8]{inputenc}
\usepackage{graphicx}
\usepackage{listings}  
\usepackage[a4paper, total={6in, 9in}]{geometry}

\title{Fighting Zika with Wolbachia}
\author{Aditi Kabra, Divyansh Garg, Dang Pham}
\date{November 2016}

\begin{document}
Dear Editor,\\

Zika is rising at an alarming rate in Puerto Rico, and we need to combat the devastating disease in the best ways we can. Since we don’t yet have any cure, our actions must be directed to prevention. Zika is spread by the \textit{Aedes} genus mosquito. We would do well to disable the vector of this disease.\\

One extremely effective way of doing this is by releasing lab-grown mosquitoes that contain the \textit{Wolbachia} bacterium. These mosquitoes are far less likely to spread Zika, and when allowed to freely mingle with the regular mosquito population, will compete with and overwhelm it. With an effective strategy for release of lab grown mosquitoes, we can soon eliminate Zika carriers.\\

The change in mosquito population pattern due to the introduction of the \textit{Wolbachia} infected mosquito naturally lends itself to mathematical modeling. To be assure that adding more mosquitoes to the environment will significantly reduce danger of being bitten by a Zika-bearing mosquitoes, we developed a model and also developed another simulation to be in check with our model.\\

Firstly, we developed a system of differential equations based on a purely theoretical framework that modeled mosquito population behavior. We solved these to find that it would be far more effective in any time period greater than a few weeks to release exclusively female \textit{Wolbachia} carrying mosquitoes into the environment.\\

We kept two milestones in mind - to reduce the number of people infected every day to half of the current rate in one year, and to ensure that less than 10,000 pregnant women were affected by Zika over the span of the five years of mosquito release. We are also aware that the population will not appreciate a large release of mosquito every month. However, comparing our solution to cities and places that have employed the same "\textit{Wolbachia}-strategy" reveals that our release is minimal. \textbf{We found that these aims could be achieved by releasing 25,500 female and 0 male \textit{Wolbachia} carrying mosquitoes per month, for 5 years} - comparing to Clovis, California which is planning to release 44,000 mosquitoes per week.\\

We also included a computer simulation to model the spread of the disease, which reaffirmed our results. Additionally, we also predict that though we may add many mosquitoes, the overall population of mosquitoes will not drastically increase.\\

We urge actions to be made immediately as our model also predicts dire consequences if no measures are taken. Specifically, the whole island will eventually be engulfed with Zika-infected human and mosquitoes.\\

Thus, assured as we are that this method of controlling mosquitoes is very effective. We encourage actions to be taken and fight fire with fire, and mosquitoes with mosquitoes, and vanquish the Pandora’s Box we call Zika before more harm can be done. \\

Sincerely,\\ \\

Aditi Kabra, Divyansh Garg, Dang Pham


\end{document}
